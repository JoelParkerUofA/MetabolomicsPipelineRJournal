% !TeX root = RJwrapper.tex
\title{MetabolomicsPipeline: An R Package for metabolomics research}


\author{by Joel Parker M.S and Bonnie LaFleur PhD}

\maketitle

\abstract{%
An abstract of less than 150 words.
}

\section{Introduction}\label{introduction}

Introduction to Metabolon:
* What is Metabolon?
* What do they provide their customers (types of analysis).

Limitations of results from metabolon.
* Metabolon uses proprietary source code
* Additional quality control
* Further exploration
* Customized analysis
* Many plots to sort through
* Subpathway hypothesis testing

Introduction to MetabolomomicsPipeline:
* Provides users with the tools for an end to end workflow for the analysis
of Metabolon data.

\begin{itemize}
\item
  Uses common data structures to allow for easy integration with preexisting
  genomic pipelines.
\item
  Pairwise comparisons for metabolites which can handle differing experimental
  designs.
\item
  Interactive plotting to help identify metabolites of interest.
\item
  R package can be utilized through an R shiny wrapper
\end{itemize}

\textbf{Picture of workflow}

\section{Background}\label{background}

Some packages on interactive graphics include \CRANpkg{plotly} (Sievert 2020) that interfaces with Javascript for web-based interactive graphics, \CRANpkg{crosstalk} (Cheng and Sievert 2021) that specializes cross-linking elements across individual graphics. The recent R Journal paper \CRANpkg{tsibbletalk} (Wang and Cook 2021) provides a good example of including interactive graphics into an article for the journal. It has both a set of linked plots, and also an animated gif example, illustrating linking between time series plots and feature summaries.

\section{\texorpdfstring{Customizing tooltip design with \pkg{ToOoOlTiPs}}{Customizing tooltip design with }}\label{customizing-tooltip-design-with}

\pkg{ToOoOlTiPs} is a packages for customizing tooltips in interactive graphics, it features these possibilities.

\section{A gallery of tooltips examples}\label{a-gallery-of-tooltips-examples}

The \CRANpkg{palmerpenguins} data (Horst, Hill, and Gorman 2020) features three penguin species which has a lovely illustration by Alison Horst in Figure \ref{fig:penguins-alison}.

\begin{figure}
\includegraphics[width=1\linewidth,height=0.3\textheight]{figures/penguins} \caption{Artwork by \@allison\_horst}\label{fig:penguins-alison}
\end{figure}

Table \ref{tab:penguins-tab-static} prints at the first few rows of the \texttt{penguins} data:

\begin{table}
\centering
\caption{\label{tab:penguins-tab-static}A basic table}
\centering
\fontsize{7}{9}\selectfont
\begin{tabular}[t]{l|l|r|r|r|r|l|r}
\hline
species & island & bill\_length\_mm & bill\_depth\_mm & flipper\_length\_mm & body\_mass\_g & sex & year\\
\hline
Adelie & Torgersen & 39.1 & 18.7 & 181 & 3750 & male & 2007\\
\hline
Adelie & Torgersen & 39.5 & 17.4 & 186 & 3800 & female & 2007\\
\hline
Adelie & Torgersen & 40.3 & 18.0 & 195 & 3250 & female & 2007\\
\hline
Adelie & Torgersen & NA & NA & NA & NA & NA & 2007\\
\hline
Adelie & Torgersen & 36.7 & 19.3 & 193 & 3450 & female & 2007\\
\hline
Adelie & Torgersen & 39.3 & 20.6 & 190 & 3650 & male & 2007\\
\hline
\end{tabular}
\end{table}

Figure \ref{fig:penguins-ggplot} shows an plot of the penguins data, made using the \CRANpkg{ggplot2} package.

\begin{verbatim}
penguins %>% 
  ggplot(aes(x = bill_depth_mm, y = bill_length_mm, 
             color = species)) + 
  geom_point()
\end{verbatim}

\begin{figure}
\centering
\includegraphics{main_files/figure-latex/penguins-ggplot-1.pdf}
\caption{\label{fig:penguins-ggplot}A basic non-interactive plot made with the ggplot2 package on palmer penguin data. Three species of penguins are plotted with bill depth on the x-axis and bill length on the y-axis. Visit the online article to access the interactive version made with the plotly package.}
\end{figure}

\section{Summary}\label{summary}

We have displayed various tooltips that are available in the package \pkg{ToOoOlTiPs}.

\section*{References}\label{references}
\addcontentsline{toc}{section}{References}

\phantomsection\label{refs}
\begin{CSLReferences}{1}{0}
\bibitem[\citeproctext]{ref-crosstalk}
Cheng, Joe, and Carson Sievert. 2021. \emph{{crosstalk}: Inter-Widget Interactivity for HTML Widgets}. \url{https://CRAN.R-project.org/package=crosstalk}.

\bibitem[\citeproctext]{ref-palmerpenguins}
Horst, Allison Marie, Alison Presmanes Hill, and Kristen B Gorman. 2020. \emph{{palmerpenguins}: Palmer Archipelago (Antarctica) Penguin Data}. \url{https://allisonhorst.github.io/palmerpenguins/}.

\bibitem[\citeproctext]{ref-plotly}
Sievert, Carson. 2020. \emph{{Interactive Web-Based Data Visualizatio}n with r, Plotly, and Shiny}. Chapman; Hall/CRC. \url{https://plotly-r.com}.

\bibitem[\citeproctext]{ref-RJ-2021-050}
Wang, Earo, and Dianne Cook. 2021. {``Conversations in Time: Interactive Visualisation to Explore Structured Temporal Data.''} \emph{The R Journal}. \url{https://doi.org/10.32614/RJ-2021-050}.

\end{CSLReferences}


\address{%
Joel Parker M.S\\
University of Arizona\\%
Department of Epidemiology and Biostatistics\\ Tucson, Arizona\\
%
\url{https://publichealth.arizona.edu/departments/epi-bio/biostatistics}\\%
\textit{ORCiD: \href{https://orcid.org/0000-0002-3411-3818}{0000-0002-3411-3818}}\\%
\href{mailto:joelparker@arizona.edu}{\nolinkurl{joelparker@arizona.edu}}%
}

\address{%
Bonnie LaFleur PhD\\
R. Ken Coit College of PharmacyBio5 Institute\\%
\\
%
\url{https://www.pharmacy.arizona.edu/person/bonnie-lafleur-0}\\%
\textit{ORCiD: \href{https://orcid.org/0000-0002-4939-9337}{0000-0002-4939-9337}}\\%
\href{mailto:blafleur@arizona.edu}{\nolinkurl{blafleur@arizona.edu}}%
}
